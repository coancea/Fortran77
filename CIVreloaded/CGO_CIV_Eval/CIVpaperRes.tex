\documentclass{sig-alternate}



\usepackage{makeidx}

%\newenvironment{titemize}{
%    \vspace{-.4\baselineskip}
%    \begin{itemize}
%    \itemsep -.1\baselineskip
%}
%{\end{itemize} \vspace{-.3\baselineskip}}
%\newcommand{\titem}{\item[$\bullet$]}
%

\usepackage{amsmath}
\usepackage{amssymb}
%\usepackage{amsthm}

\usepackage{multicol}
\usepackage{multirow}
\usepackage{graphicx}
\usepackage{color}
\usepackage{subfigure}
\usepackage{epsfig}
\usepackage{fancybox}
\usepackage{fancyvrb}

\usepackage{rotating}

\usepackage{hyperref} % comes last, as it redefines a couple of commands
\hypersetup{
    colorlinks,%
    citecolor=black,%
    filecolor=black,%
    linkcolor=black,%
    urlcolor=blue
}

\DefineVerbatimEnvironment{colorcode}%
        {Verbatim}{fontsize=\small,commandchars=\\\{\}}   % \normalsize
        %{Verbatim}{fontsize=\scriptsize,commandchars=\\\{\}}


\definecolor{DikuRed}{RGB}{130,50,32}
\newcommand{\emp}[1]{\textcolor{DikuRed}{ #1}}
\definecolor{CosGreen}{RGB}{10,100,70}
\newcommand{\emphh}[1]{\textcolor{CosGreen}{ #1}}


\newcommand{\mymath}[1]{$ #1 $}
\newcommand{\myindx}[1]{_{#1}}
\newcommand{\myindu}[1]{^{#1}}
\newcommand{\mymathbb}[1]{\mathbb{#1}}


\hyphenation{in-de-pen-dence}
\hyphenation{pa-ra-lle-lize}
\hyphenation{sub-scrip-ted}
\hyphenation{sub-scrip-ted}
\pagestyle{plain}


\begin{document}

%\conferenceinfo{ICS}{'13 Eugene, Oregon USA}


%\title{ Summarizing Accesses Without Closed-Form Solutions in Loop : Beyond Idiom Recognition }   
\title{CGO'15 ARTIFACT EVALUATION FOR PAPER:\\Scalable Conditional Induction Variables (CIV) Analysis}   

\numberofauthors{2}

\author{
\alignauthor
Cosmin E. Oancea\\
       \affaddr{Department of Computer Science}\\
       \affaddr{University of Copenhagen}\\
       \email{cosmin.oancea@diku.dk}
% 2nd. author
\alignauthor
Lawrence Rauchwerger\\
       \affaddr{Department of Computer Science and Eng.}\\
       \affaddr{Texas A \& M University}\\
       \email{rwerger@cs.tamu.edu}
}


\maketitle
%\thispagestyle{empty}


\pagenumbering{arabic}

\begin{abstract}

Subscripts using variables, named {\sc civ}, that cannot be expressed 
as a formula in terms of the enclosing-loop indices,  
appear in the low-level implementation of common programming abstractions 
such as filter operations or stacks or vectors in which elements are 
inserted at the end, and pose significant challenges to automatic
parallelization.

This paper presents a flow-sensitive technique that summarizes 
both {\sc civ}-based and affine subscripts to program level,
and under the same representation: Our technique (i) requires no 
modifications of the dependency test, which is agnostic to the 
original shape of the subscripts, and (ii) may prove useful
in a wider context, e.g., array {\sc ssa}. 
%, and may be applied in a wider context, e.g., array {\sc ssa}.
%
%On the one hand, our technique does not require modifications of 
%the dependency test, which is agnostic to the original shape of 
%the subscripts. 
In comparison, related work uses specialized dependency tests that 
%operate at the level of each pair of read-write subscripts, and
disambiguate each pair of read-write accesses (locally), and 
have been found less-suited than the summary-based ones in analyzing 
more complex loops. 
%than the ones based on summaries.   
%On the other hand, our technique is not restricted to 
%dependency tests and may be applied for example in the context 
%of array {\sc ssa}.
%
We report a systematic evaluation of five Fortran benchmarks that demonstrates 
  (i) efficient (scalable) parallel execution, 
 (ii) that important loops exhibiting {\sc civ} subscripts are not infrequent, and 
(iii) that our techniques covers loops that were either not reported or were 
        previously solved each with a different technique.
\end{abstract}


\category{D.1.3}{Concurrent Programming}{Parallel Programming}
\category{D.3.4}{Processors}{Compiler}


\terms{Performance, Design, Algorithms}

\keywords{
array-reference summaries, conditional induction variables}

\newpage
\section{Introduction} 

This document instructions for the CGO'15 artifact evaluation for accepted paper: 
``Scalable Conditional Induction Variables (CIV) Analysis'', by Cosmin E. Oancea 
and Lawrence Rauchwerger.

{\bf The paper submitted to CGO'15, and this document in pdf and text form
are available at:}\\ 
{\tt~~~~http://www.diku.dk/$\sim$zgh600/CGO\_CIV\_Eval.tar.gz}
\vspace{1ex}

WE HAVE PREPARED A MACHINE WHERE THE REVIEWER CAN LOG IN TO
TEST THE RESULTS; Please contact {\bf GRIGORI FURSIN} to get the
associated information (ACCOUNT and PASSWORD).

In order to do the actual testing the reviewer is kindly asked to
log in to said machine and to follow the instructions in this document.

\section{Requirements}

Our submission assumes a Linux {\tt bash} environment, supporting: 
{\tt make}, {\tt sed} (stream editor), {\tt gcc}, {\tt g++}, {\tt gfortran} 
with support for \textsc{OpenMP} directives, i.e., {\tt -fopenmp} compilation flag.
We have tested our artifacts under several versions of {\tt Ubuntu} and {\tt Suse}, 
and several versions of {\tt gcc}, e.g., {\tt gcc 4.8.2 (Ubuntu)}, {\tt gcc 4.6.4, gcc 4.7.2 (Suse)}.  


\section{Installation in Three Steps}

If not already decompressed, {\em the first step} is to decompress this archive:\\
$\mbox{ }${\tt~~~~\$ tar zxvf POLARIS\_CGO.tar.gz}\vspace{1ex}

This will create directory {\tt POLARIS} which contains:\\
$\mbox{ }${\tt~~~~\$ cd POLARIS; ls}\\
$\mbox{ }${\tt~~~~CGO\_Eval  Makefile  README  setPolarisEnv.sh  trunk}\vspace{1ex}

Folder {\tt trunk} contains our source-to-source Fortran77 compiler,
a modified version of Polaris research compiler.
Folder {\tt CGO\_Eval} contains the 6 benchmarks reported in the paper,
    each corresponding to one of the subfolders: {\tt Bdna}, {\tt Nasa7},
    {\tt PriceRec}, {\tt PriceInd}, {\tt Track}, {\tt Tree}.

{\em The second step} is to set some of the environment variables
that are needed to run the compiler and link with runtime 
libraries, e.g., this will modify {\tt \$PATH} and {\tt \$LD\_LIBRARY\_PATH}:\\
$\mbox{ }${\tt~~~~POLARIS\$ source setPolarisEnv.sh}\vspace{1ex}

Please test that {\tt \$POLARIS\_ROOT} and {\tt \$POLARIS\_RUNTIME} are set:\\
$\mbox{ }${\tt~~~~POLARIS\$ echo \$POLARIS\_ROOT; echo \$POLARIS\_RUNTIME}\\
$\mbox{ }${\tt~~~~path\_to\_POLARIS/trunk}\\
$\mbox{ }${\tt~~~~path\_to\_POLARIS/trunk/runtime/rtlibs}.\vspace{2ex}

{\em The third step} {\tt~~~~POLARIS\$ make}
\begin{itemize}
    \item[1]  builds and installs our compiler (modified Polaris), 
    \item[2]  compiles the $6$ benchmarks reported in the paper with
             our source-to-source compiler 
             which takes sequential Fortran77 code and generates 
             OpenMP-annotated (parallel) Fortran77 code.
             {\em It might take 1 hour or longer.}
    \item[3] compiles the resulted parallel Fortran code
              with the {\tt gfortran} compiler ({\tt -fopenmp} flag) 
                and links with the runtime libraries.
    \item[4] and finally runs the benchmarks.
\end{itemize}

\section{A Closer Look at the Makefile}

Shows that there are several targets:
\begin{itemize}
    \item {\bf \$make polaris} will build our source-to-source compiler.
          (It is enough to do this step once.)

    \item {\bf \$make ref\_out\_benches} will create the reference output for each benchmark.
          (It is enough to do this step once.)
          This is due to the fact that switching the gfortran version and/or hardware
            may slightly alter results. Since we do not want to complicate matters,
            e.g., to require installation of SPEC's diff, we choose to validate
            results against the output of the original Fortran program.
          Validation simply checks that the output file of the parallel version
            (and the sequential version with profiling) is identical with
            the reference output. This holds on all tested configurations.
          Reference output can be created for each individual benchmark, e.g.,\\
            $\mbox{ }${\tt~~~~\$ cd CGO\_Eval/Bdna; make ref\_out; cd ../..}

    \item {\bf \$make build\_benches} will compile the $6$ benchmarks reported in the paper
            with our compiler generating sequential (suffixed by {\tt sp}) and 
            parallel (suffixed by {\tt ha}) source-code versions that are automatically 
            instrumented with profiling information.
            (It is enough to do this step once.) 
            {\em Note that this step might take about one hour; benchmark 
            {\tt Track} takes the longest to build: $\sim 45$ minutes}.
          Benchmarks can be also built individually, e.g.,\\
             {\tt~~~~\$ cd CGO\_Eval/Bdna; make buildSP; make buildHA; cd ../..}

    \item {\bf \$make run\_ha\_benches} compiles with {\tt gfortran -O3 -fopenmp} and 
            runs the parallel version of the code. Timing and validation results
            are generated and can be inspected in folder  
            {\tt CGO\_Eval/BenchName/TimingRes/}, where {\tt BenchName} is one of the
            six benchmark names. This is described in detail in (next) Section~\ref{sec:TimeValid}.
          Benchmarks can be also run individually, e.g.,\\
             $\mbox{ }${\tt~~~~\$ cd CGO\_Eval/Bdna; make runHA; cd ../..}

    \item {\bf \$make run\_sp\_benches} is similar to {\tt run\_ha\_benches}, but it runs the
            sequentially profiled version of the code and generates validation and 
            timing results.

    \item {\bf \$make clean} cleans up {\em everything}: our source-to-source compiler,
            the generated parallel source-code for each benchmark, and 
            any other intermediate files that have been produced.
            Each benchmark can be cleaned individually as well:\\
            $\mbox{ }${\tt~~~~cd CGO\_Eval/Bdna; make clean; cd ../..}
\end{itemize}

Finally, runtime results on a different number of processors can be obtained
by setting manually the {\tt OMP\_NUM\_THREADS} environment variable:\\
$\mbox{ }${\tt~~~~\$ export OMP\_NUM\_THREADS=2}\\
\noindent and running the benchmarks again:\\
$\mbox{ }${\tt~~~~\$ make run\_ha\_benches}\\
to get the runtime on 2 cores, for example. 
Note that the previous timing results will be overwritten.

\section{Timing and Validation}
\label{sec:TimeValid}

Timing and validation results can be consulted in the following files:\\
$\mbox{ }${\tt~~~~POLARIS\$ ls CGO\_Eval/BenchName/TimingRes/}\\
$\mbox{ }${\tt~~~~timing\_ha.out  timing\_sp.out  VALID\_ha  VALID\_sp}\\
\noindent where {\tt BenchName} is one of the becnhmarks reported in the paper:
{\tt Bdna}, {\tt Nasa7}, {\tt PriceRec}, {\tt PriceInd}, {\tt Track}, {\tt Tree}.

If the execution is valid then {\tt VALID\_ha} or {\tt VALID\_sp} files appear.
Otherwise, {\tt INVALID\_ha} or {\tt INVALID\_sp} files appear. 

\begin{figure*}[t]
\begin{colorcode}
POLARIS\$ more CGO\_Eval/PriceRec/TimingRes/timing\_sp.out
             Loop name      Category   Count      Time       \%       Granularity
=================================================================================
             WHOLE\_PGM   CT:Sequential    1     1.064698   100.00      1.064698
MONTECARLOPRICING\_do10   CT:Sequential    1     1.064661   100.00      1.064661



POLARIS\$ more CGO_Eval/PriceRec/TimingRes/timing\_ha.out
                                 Loop name          Category   Count  Time        \%    Granularity
========================================================================================================
                                 WHOLE_PGM        CT:Sequential  1   0.351401   100.00   0.351401
                    MONTECARLOPRICING_do10     RT:Par/Inspector  1   0.327876    93.31   0.327876
CIV_COMP_SOBOLACC_MONTECARLOPRICING_DO10_HL  Pass:Insp.Overhead  1   0.023480     6.68   0.023480
\end{colorcode}
%\hrule
\caption{Profiling of the sequential ({\tt timing\_sp.out}) and parallel ({\tt timing\_ha.out}) execution of benchmark {\tt PriceRec} run on some $4$-core machine.}
\label{fig:ListingHAandSP} %
\end{figure*}


\begin{itemize}
    \item[SP] Suffix {\tt sp} refers to the sequential version supporting 
                profiling of the loops specified in\\
                {\tt CGO\_Eval/BenchName/support/ipa\_framework.loops}\\
                and reported in the paper, while
    \item[HA] Suffix {\tt ha} refers to the parallel version of the code with 
                automatic profiling instrumentation for (i) the orginal (parallelized) 
                loops, and (ii) for whatever runtime predicates/inspectors were 
                generated in order to prove parallelism.  The latter are overhead.
\end{itemize}

For example, Figure~\ref{fig:ListingHAandSP} shows the profiling information
for the sequential (top) and parallel (bottom) versions of benchmark {\tt PriceRec}:
\begin{itemize}
    \item The {\em sequential} benchmark-level runtime is $1.064698$ and loop 
        {\tt MONTECARLOPRICING\_do10} accounts for $\sim 100\%$ 
        of the total sequential runtime. 

    \item The {\tt parallel} benchmark-level runtime (on some $4$-core machine) is 
            $0.351401$, and the parallelized loop takes $\sim 93.31\%$ 
            of the total paralle runtime. The remaning $6.68\%$ of runtime is used 
            in inspector\\ {\tt CIV\_COMP\_SOBOLACC\_MONTECARLOPRICING\_DO10\_HL},\\
            which corresponds to the parallel slice of the original loop that computes the
            CIV values. The latter is necessary for the parallelization of the original loop,
            please see paper for more details.
\end{itemize}



\section{What is to Verify?}

One can verify
\begin{itemize}
    \item[1] That the loops reported in the paper, e.g., Table~$1$ and Figure~$10$, 
        are recognized as parallel, run in parallel (see speedups w.r.t. sequential 
        version) and validate.

    \item[2] That the loop-slices that compute the {\sc civ} values, and in general
                the other runtime tests that enable paralleism, add in most
        cases negligible overhead with the exception of benchmarks:
        \begin{itemize}
            \item  {\tt PriceRec} where overhead is still small $\sim 6.5\%$ of the
                    total parallel runtime, and
            \item {\tt Track} where overhead accounts for $\sim 50\%$ of the total
                parallel runtime.
        \end{itemize}

    \item[3] That they give good speedup w.r.t. the sequential version. 
        The results published in the paper were obtained on different machine(s),
        hence we do not expect the current speedups to be exactly as in the paper.
        However we expect them to be relatively closed to those:
        some better, some worse than the ones reported in the paper on 8 cores.
\end{itemize}

       Note that the bottom-right graph of Figure~$10$ in the paper was obtained
        on a ``Better-Bandwidth PowerPC P5+'' machine, precisely because on other
        commodity machines we were dissapopinted with the scalability of {\tt Track}. 
        Hence the results on the tested configuration might not come close to 
            the ones reported in the bottom-right graph of Figure~$10$. 

\section{Limitations of the Approach}

{\bf First}, building the parallel (and sequentially profiled) version of the code
            for benchmarks {\tt PriceInd}  and {\tt PriceRec} requires hand-based
            modification of the code, mostly because our compiler (Polaris) 
            does not recognize operators such as {\tt XOR}, {\tt SHIFTL}, {\tt SHIFTR}, etc.

       This was reported in the paper (page 9):
            ``Because our implementation does not support {\tt XOR}, we have manually 
                turned the {\tt XOR} into addition before compilation and then back 
                to {\tt XOR} after compilation.'' 
        For the purpose of automatic testing we have automated these transformations
        using {\tt sed} stream editor, in the {\tt make buildSP} and {\tt make buildHA}
        targets of benchmarks {\tt PriceInd} and {\tt PriceRec}.

{\bf Second}, on some configurations, {\tt make buildHA} gives a segmentation fault
        on benchmark {\tt Track}, at the final memory deallocation. The parallel
        version is however produced before it, and {\em everything still works correctly}.
        We have not been able to find yet the bug, and do not know whether it is ours or not. 

%\acks
%\vspace*{-1ex}
\section{Acknowledgements}
We would like to thank the reviewers for their hard work in verifing these artifacts.
It is deeply appreciated!





%%\bibliographystyle{abbrvnat}
%\bibliographystyle{abbrv}
%%\softraggedright
%\begin{small}
%\bibliography{CIVpaper}
%\end{small}


\end{document}


%%%%%%%%%%%%%%%%%%%%%%%%%
%%%% USELESS STUFF %%%%%%
%%%%%%%%%%%%%%%%%%%%%%%%%


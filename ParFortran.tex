\documentclass[prodmode,acmtecs]{acmsmall}

% Package to generate and customize Algorithm as per ACM style
\usepackage[ruled]{algorithm2e}
\renewcommand{\algorithmcfname}{ALGORITHM}
\SetAlFnt{\small}
\SetAlCapFnt{\small}
\SetAlCapNameFnt{\small}
\SetAlCapHSkip{0pt}
\IncMargin{-\parindent}

% Metadata Information
%\acmVolume{9}
%\acmNumber{4}
%\acmArticle{39}
%\acmYear{2010}
%\acmMonth{3}

% Document starts
\begin{document}

% Page heads
\markboth{C. Oancea and L. Rauchwerger}{A Hybrid Approach to Loop Parallelization}

% Title portion
\title{A Hybrid Approach to Loop Parallelization}
\author{Cosmin E. Oancea
\affil{University of Copenhagen}
Lawrence Rauchwerger
\affil{Texas A \& M University}}


\begin{abstract}

\end{abstract}

\category{D.1.3}{Concurrent Programming}{Parallel Programming}
\category{D.3.4}{Processors}{Compiler}


\terms{Performance, Design, Algorithms}

\keywords{auto-parallelization, array-reference summaries, conditional induction variables ({\sc civ}).}

\acmformat{Cosmin E. Oancea and Lawrence Rauchwerger, 2014.
A Lightweight Hybrid Approach to Loop Parallelization.}


\begin{bottomstuff}
Author's addresses: 
Cosmin E. Oancea, Department of Computer Science,
University of Copenhagen, cosmin.oancea@diku.dk; 
Lawrence Rauchwerger, Department of Computer Science and Engineering,
Texas A \& M University, rwerger@cs.tamu.edu.
\end{bottomstuff}

\maketitle


\section{Introduction}


\section{Preliminaries}

\subsection{Motivating Example}

\subsection{Linear-Memory-Access-Descriptors (LMAD) Summaries}

\subsection{Unified-Set-Reference (USR) Summaries}

\subsection{Summary Construction to Program Level}

\subsection{Loop Independence Equations}


\section{Basic Translation From Summary To Predicate Language}

\subsection{Extracting Predicates from LMADs}

\subsection{Elementary Logical Inference Rules}

\subsection{Slicing and Common-Path Optimization of Predicates}


\section{Predicates for Non-Affine Subscripts}
    This is described in~\cite{SummaryMonot}.

\subsection{Motivation, Problem Statement and Simple Example}

\subsection{When, Where and How to Exploit Monotonic Summaries}

\subsection{Summaries with Quasi-Linear Subscripts}

\subsection{Summaries with Quadratic Subscripts}

\subsection{Summaries with Indirect-Array Subscripts}

\subsection{Overall Design Strategy}


\section{Summarizing Across Conditional Induction Variables}

\section{Related Work}


\section{Conclusions}



% Appendix
\appendix
\section*{APPENDIX}
\setcounter{section}{1}
In this appendix, we ...

\appendixhead{ZHOU}

% Acknowledgments
\begin{acks}
The authors would like to thank ...
\end{acks}

% Bibliography
\bibliographystyle{ACM-Reference-Format-Journals}
\bibliography{ParFortran}
                             % Sample .bib file with references that match those in
                             % the 'Specifications Document (V1.5)' as well containing
                             % 'legacy' bibs and bibs with 'alternate codings'.
                             % Gerry Murray - March 2012

% History dates
%\received{February 2007}{March 2009}{June 2009}

% Electronic Appendix
\elecappendix

\medskip

\section{This is an example of Appendix section head}

Lalala1


\section{Appendix section head}

Lalala2

\end{document}
% End of v2-acmsmall-sample.tex (March 2012) - Gerry Murray, ACM


